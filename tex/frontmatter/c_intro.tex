\chapter*{Introduction}

En l'an 2016, Linux a fait une montée de popularité très remarquable au niveau
international. Jusqu'à Microsoft qui intègre une sous-couche d'Ubuntu dans son
système Windows 10, la croissance des ventes des appareils Android (qui utilisent
le noyau Linux), la popularité du nano-ordinateur RaspberryPi qui tourne
initialement à l'aide de Raspbian (un système dérivé de Debian 8), la soudaine
popularité de Linux Mint (un système dérivé d'Ubuntu)... Bref, Linux devient de
plus en plus populaire au niveau international.

Les linuxiens de Madagascar en sont conscients et ils en sont contents, mais
cette popularité n'arrive pas jusqu'à nos portes. Pour pas mal d'autres
utilisateurs d'ordinateurs lambda, Linux ne leur sert qu'à supprimer les virus
foza orana des clés USB, à pirater (ceux qui ont juste eu vent de Kali Linux),
et à créer un serveur (des dev web débutants qui ont peut-être appris à tirer
des informations concernant le serveur distant sur lequel ils hébergent leurs
sites?). Peu l'utilisent encore en tant que système d'exploitation principal pour
le travail, les loisirs, le plaisir de s'amuser dessus... Comme il était écrit
sur un certain wallpaper sur lequel je suis tombé :

\begin{quotation}
  We tell people we use Linux because it's secure. Or because it's free, because
  it's customizable, because it's free (the other meaning), because it has
  excellent community support...

  But all of that is just marketing bullshit. We tell that to non-Linuxers because
  they wouldn't understand the real reason. And when we say those false reasons
  enough, we might even start to believe them ourselves.

  But deep underneath, the real reason remains.

  We use Linux because it's fun.

  It's fun to tinker with your system. It's fun to change all the settings,
  break the system, then have to go recovery mode to repair it. It's fun to have
  over a thousand distros to choose from. It's fun to use the command line.

  Let me say that again. It's fun to use the command line.

  Non wonder non-Linuxers wouldn't understand.

  The point with Linux fans is we use Linux for its own sake. Sure, we like to
  get work done. Sure, we like to be secure from viruses. Sure, we like to save
  money. But those are only the side effects. What we really like is playing
  with the system, poking around, and discovering fascinating facts about the
  software that lies underneath it.
\end{quotation}

La traduction :

\begin{quotation}
  On dit aux gens qu'on utilise Linux parce que c'est secure. Ou bien que c'est
  gratuit, personnalisable, libre, il y a une bonne entraide entre la communauté...

  Mais tout cela, c'est du marketing. On dit cela aux non-linuxiens car ils ne
  comprendraient pas la vraie raison. Et à force de dire ces fausses raisons, on
  pourrait y croire nous-mêmes.

  Mais la vraie raison est bien plus profonde.

  On utilise Linux parce que c'est amusant.

  C'est amusant de titiller le système. C'est amusant de modifier tous les
  reglages, de casser le système, et de devoir passer en mode recovery pour le
  réparer. C'est amusant d'avoir plus de mille distro pour avoir l'embarras du
  choix. C'est amusant d'utiliser la ligne de commande.

  Je répète, c'est amusant d'utiliser la ligne de commande.

  Ce serait normal que les non-linuxiens ne puissent comprendre.

  Le truc des fans de Linux, c'est qu'on l'utilise pour son propre compte. Oui,
  on aime que le travail soit fait. Oui, on aime être protégés des virus. Oui,
  on aime économiser de l'argent. Mais ce ne sont là que des effets secondaires.
  Ce qu'on aime vraiment, c'est jouer avec le système, toucher à tout, et
  découvrir des faits fascinants à propos du logiciel qui touchent à ses profondeurs.
\end{quotation}

Je pense que cette philosophie résume bien ce que les membres les plus actifs de
la communauté Gasy Tia GNU/Linux pensent du système. Et si vous voulez ressentir
cette sensation, vous êtes plus que prêts à poursuivre la lecture. Sinon, les
effets secondaires sont valides.

Nous pouvons donc commencer notre apprentissage.

% Moyens techniques utilisés pour la rédaction
\chapter*{Détails techniques}

\paragraph{Langages} ~\\

Cet ouvrage a été rédigé en utilisant \LaTeX{}, un langage utilisé en extension
du langage \TeX{} qui sert surtout à la création de publications scientifiques.

Ce langage a été choisi pour que l'ouvrage soit un « Openbook » auquel
ceux qui pourront partager leurs connaissances puissent contribuer. Ceci parce
qu'évidemment, les connaissances d'un seul individu ne seront pas suffisantes
pour donner les informations nécessaires.

\paragraph{Système de gestion de versions} ~\\

Nous utilisons \emph{Git} principalement pour pouvoir collaborer les uns avec
les autres.

Nous l'avons choisi car il a été créé par Linus \bsc{Torvalds}, créateur du
noyau Linux, donc c'est sans doute celui qui est le mieux intégré au système
dont nous allons le plus parler dans ce livre.

\paragraph{Éditeur de texte} ~\\

Je suppose qu'on aura une liste ici, car chaque collaborateur utilise surement
l'éditeur qui lui est le plus familier.

On se passera peut-être de descriptions.

\begin{enumerate}
  \item Atom
\end{enumerate}

% Chap 2 - Historique
\chapter{Historique}

\section{Unix}

Au commencement, il y avait \emph{Unix} considéré à l'époque comme le meilleur
système d'exploitation en activité. Assez compliqué à utiliser, donc utilisé
uniquement par les informaticiens professionels de l'époque.

À l'époque, les ordinateurs ne pouvaient pas faire mieux que du texte blanc sur
fond noir, en gros, une console où on entrait des commandes.

\section{Le projet GNU}

Vers 1984, Richard \bsc{Stallman} créa le \emph{Projet GNU}, une copie d'
\emph{Unix}, pour la simple raison que ce dernier était payant et devenait de
plus en plus cher.

Il a donc développé les programmes de base d'un système d'exploitation, comme
la copie de fichiers, les attributs de ceux-ci, et tant d'autres.

En complément de ces programmes, il devait y avoir le \emph{GNU Hurd}, noyau
que devait utiliser le système.

\section{Linux}

En 1991, Linus \bsc{Torvalds} s'est amusé « en tant que hobby » à créer un noyau
de système d'exploitation tout d'abord nommé \emph{MINIX} avant de prendre le
nom \emph{Linux} un peu plus tard.

Comme \emph{GNU Hurd} avait du mal à aboutir, Richard \bsc{Stallman} et Linus
\bsc{Torvalds} se sont donc associés pour former le système d'exploitation
\emph{GNU/Linux}.

Il faut comprendre que les programmes sur \emph{Unix} et \emph{GNU/Linux} ne sont
pas les mêmes, ils n'ont pas été compilés à partir du même code source. Ils
fonctionnent juste de la même manière.

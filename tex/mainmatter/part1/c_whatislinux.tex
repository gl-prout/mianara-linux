% Chap 1 - Qu'est-ce que linux?
\chapter{Qu'est-ce que Linux ?}

\section{Système d'exploitation}

Et j'ai tout spoilé dans le titre. Enfin, pas tout à fait, mais je l'ai spoilé.
En fait, la véritable appelation du système d'exploitation est \emph{GNU/Linux}
mais pour faire plus court, on dit Linux.

Donc, Linux est un système d'exploitation, la fusion du \emph{Projet GNU} et du
noyau \emph{Linux} (on y reviendra), qui s'utilise presque de la même manière
que les systèmes \emph{Unix}.

\section{Un noyau}

Tout système d'exploitation est basé sur un noyau ou \emph{kernel} en anglais.
Les systèmes d'exploitation GNU/Linux (je réutilise ce terme ici pour qu'il n'y
ait pas confusion) sont donc basés sur le noyau Linux, un noyau de système d'
exploitation créé par Linus \bsc{Torvalds} et utilisé initialement par
Richard \bsc{Stallman}, initiateur du Projet GNU.

Il faut savoir que dans ce livre, on étudiera le système, pas le noyau. Si ce
projet d'Openbook aboutit, il y aura peut-être des auteurs qui vont écrire un
livre sur le noyau, mais nous, on se concentrera sur le système. Comme ça, c'est
dit.

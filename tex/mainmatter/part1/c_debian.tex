% Chap 3 - Debian
\chapter{Et qu'est-ce que Debian?}

\section{La distribution Debian}

Comme indiqué dans le titre, \textsc{Debian} est une distribution Linux, à savoir
une manière de présenter le système. Et vous comprendrez par là qu'il existe
une multitude de distributions Linux existantes qui se différencient surtout
par leurs gestionnaires de paquets par défaut et les applications préinstallées
avec le système.

Juste pour que vous connaissiez des noms, il y a par exemple les distributions
\textsc{Red Hat Linux}, \textsc{Gentoo}, \textsc{Suse}, et surtout, celle que
nous allons voir en détails: \textsc{Debian}.

À titre d'information, à partir de distributions éxistantes, des développeurs
peuvent créer d'autres distributions, et ce cycle peut se reproduire encore
plus loin, nous avons donc par éxemple la distribution mère \textsc{Debian},
à partir de laquelle la célèbre distribution \textsc{Ubuntu} a été créée, et
à partir d'Ubuntu, la distribution \textsc{Linux Mint} a également été créée.

\section{L'instigateur}

Le "papa" de Debian portait le nom de Ian \textsc{Murdoc}, son épouse porte le nom
de Debra, donc, en combinant leurs prénoms, ça a donné \textsc{Debian}.

\section{Le gestionnaire de packages}

La plupart du temps, ce qui fait la plus grande différence d'une distribution à
une autre, c'est le gestionnaire de packages principal utilisé par celui-ci.

Debian utilise \textsc{APT}, qui lui permet d'installer des packages \textsc{*.deb},
en utilisant les applications \textsc{apt-get} ou \textsc{aptitude} selon les
préférences de chacun.

Il y a également ceux qui préfèrent les IHM comme \textsc{Synaptic}. Il est à noter
que ces outils sont également sur les distributions "filles" de Debian, donc les
indications présentes ici sont également valables pour les distributions filles.

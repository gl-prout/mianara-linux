% Chap 2 - Gnome
\chapter{Gnome}

\section{Gnome}

Non, je ne me suis pas répété, \emph{Gnome} est un DE construit par dessus le
Gnome Shell. Il en est actuellement à la version 3, avec un design moderne qui
rappelle un peu l'apparence d'un bureau Mac (sans le dock) couplé à \emph{Unity},
un DE qui était très utilisé par la distribution \emph{Ubuntu}, qui est revenu
vers Gnome récemment en tant que DE par défaut.

Gnome 3 possède deux modes, le mode par défaut et le mode classique, qui
ressemble de loin au layout par défaut de son prédécesseur Gnome 2.

\section{Mate}

Prononcé "ma-té", \emph{Mate} est un fork de Gnome 2 et donne cette impression de nostalgie
chez les utilisateurs les plus anciens. Même les applications livrées avec le DE
sont les mêmes que pour Gnome 2, à savoir Nautilus, Empathy, gedit... Jusqu'aux
gadgets des barres d'outils.

\section{Cinnamon}

Avec l'arrivée de la distribution \emph{Mint}, le DE \emph{Cinnamon} est né, et
figure parmi les propositions de DE par défaut à l'installation de votre Debian
à partir de la version 8. La liste des applications par défaut est un mélange
entre ceux de Mate et de Gnome.

\section{Remarque}

Selon une expérience vécue, lorsque j'ai installé Debian Jessie, j'ai constaté
un bug d'installation lorsque j'essayais d'installer n'importe lequel des DE
utilisant le Gnome shell. Je ne sais pas si ce bug est résolu pour Debian Stretch
(la version 9), mais au cas où, je vous recommanderais plutôt d'installer un DE
basé sur \emph{X} à l'installation de Debian. Vous pourrez toujours installer
n'importe lequel des trois DE cités précédemment une fois votre système installé.

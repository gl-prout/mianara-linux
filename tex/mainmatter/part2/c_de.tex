% Chap 1 - Desktop Environment
\chapter{Desktop Environment}

\section{Mode graphique}

Dans cette partie du livre, on va se focaliser sur l'environnement graphique de
Debian. Je ne pense pas avoir à expliquer ce qu'est un environnement graphique,
car si vous lisez ceci en PDF, vous êtes surement en train d'utiliser un
environnement graphique, quel que soit l'OS que vous utilisez.

\section{Définition}

Tout comme il existe un grand nombre de distributions Linux, il existe également
un grand nombre de \emph{Desktop Environments} ou \emph{Environnements de bureaux}
que vous pouvez installer sur votre système. Et le bonus, vous pouvez en installer
plusieurs et basculer de l'un à l'autre selon votre humeur ou vos préférences.

Définissons à présent ce qu'est un \emph{DE} (pour faire court).

Un DE comme son nom l'indique est un ensemble d'applications qui partagent un
même GUI, parfois décrit comme un \emph{shell graphique}.

Un DE est composé d'icônes, de fenêtres, de barres d'outils, de répertoires, de
wallpapers et parfois de widgets graphiques.

Les shells graphiques les plus populaires sont \emph{Gnome}, \emph{K} et \emph{X}.

Dans les prochains chapitres, nous verrons quelques DE qui ont été développés
par dessus ces shells.
